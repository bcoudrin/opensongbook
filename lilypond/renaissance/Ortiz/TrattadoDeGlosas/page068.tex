%page 68 du PDF original
Dueu prima elegersi e pigliarsi quel Madrigale, o Mottetto, ouer altra opra che si voglia sonare, e poi ponersi nel Cimbalo come ordinariamente si suole porte. E colui che suona il Violone puo sopra cotal componimento sonar due, tre, e piu varietadi : e qui ne pono quattro sopra un Madrigale per essempi. La prima fia sopra il medesimo contrabasso del Madrigale con alcuna aggiunta \& alcuni larghi passaggi. La seconda fia col soprano fiorizzato e diminuito, e questo modo di sonare sara pui deletteuole e gratiofo, quando nel Cimbalo non si suoni il detto soprano. La terza fia l'accompagnamento della prima : quantunque sia piu difficile a sonarsi, per che richiede mano pui sciolta. La quarta fia con una quinta voce ouer parte aggiunta, alla quale non e obligato il sonatore che non habbia buona praticca, & habilitade di comporte.
