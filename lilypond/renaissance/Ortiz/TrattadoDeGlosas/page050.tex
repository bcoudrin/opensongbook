% page 50 du PDF, première page du livre 2
% lettrine sur le I

In questo secondo Libro fi trattano le varie maniere che fi debbiane sonare col Violone, e col Cimbalo insieme, Tresonno li maniere di sonare. La Prima si dice Fantasia. La Seconda sopra canto Piano. La terza sopra compositione di molte voci. La Fantasia non fi puo mostrare, che ciaseuno buon sonatore la suona di sua testa e di suo studio \& uso ma ben diro quel che fi richieda per sonala. La fantasia che sonera il Cimbalo sia di consonanze ben ordinare oue poi \b{entri ou encri} sonando il Violone con alcuni leggiadri passaggi, e quendo el Violone si trattiene in alcune tirare ouero archate piane allhora il Cimbalo gli risponda a proposto \& insieme faccino alcune fughe belle hauendo risguardo e risperro l'un all'altro, come suol hauersi nelli Contraponti di consierto e cosi l'uno conoscera l'altro, e con l'essercitatione commune si scopriranno si molti escellenti e degni secreti che si contengono in questa maniera di sonare di Fantasia ma delle due altre maniere si fara mentione nelli lor conueneuoli e proprii lochi.

% second chapitre
% lettrine sur le S
% V odo = peut-être "à la quinte"
\chapter{L'ordine che se ha da tener 'in accodar' il Violone col Cimbalo}
Sono molte maniere di accorar'il Violone col Cimbalo, perche si puo sonare per qual si voglia tuono, alzando o calando nel sonare un punto o piu secondo il tuono del Cimbalo ricerca, il che quantunque sia difficile, col essercitio continuo se rendera facile, pero la piu facile \& miglior maniera di accordar il Violone col Cimbalo e che quinta del Violone in \b{V odo}, sia unisono col G amaut del Cimbalo, per che a questo modo participano egualmente delli Bassi \& Alti \& in questo modo de temperamento se ha da sonar' tutto quello che se scriuera di questi Instrumenti.
Queste quatro ricercate che qui seguono mi parue di porle libere \& sciolce per essercitar la mano \& in parte \b{lar} qualche noticia del discurso che se ha da tener' quando se sonara un Violon solo.
